\chapter{LITERATURE SURVEY} % Main chapter title

\label{Chapter2} % For referencing the chapter elsewhere, use \ref{Chapter1} 


\section{Air Quality Monitoring }

Environment monitoring is crucial and necessary task in enabling healthy living of mankind. Environment monitoring is critical to know whether the quality of our environment is getting better or worse. Information gathered through environment monitoring is important to many decision makers, inside and outside the government. With the results of monitoring, the government can make informed decision about how the environment will affect people and how people are affecting the environment. A lot more work has been done in the area of air quality monitoring system in past years which is summarised below: -

\subsection{Indoor Air Quality Monitoring}
In recent years, indoor air quality (IAQ) has drawn considerable attention in both the public and scientific domains, due to the fact that most buildings appear to fall far short of reasonable air quality goals. Statistics from the U.S. Environmental Protection Agency(EPA) indicate that, on average, the indoor levels of pollutants are two to five times higher than outdoor levels and people in the U.S. spend about 90\% of their time indoors. Bad indoor air quality influences human health, safety, productivity, and comfort. IAQ is important and different people have different exposure to pollutants. Providing personalized IAQ information has the potential to increase public awareness of the relationship between their behaviour and air quality; help people to improve their living environments; and also provide valuable information to building managers, policy makers, health professionals, and scientific researchers.
\\
\\
Indoor air quality monitoring is necessary as sometimes we find that indoor level of pollution is higher than outdoor pollution level. This is because of low ventilation and cooking and heating processes. Some of the Indoor Pollution work has been surveyed and their gist is given below.

\subsubsection{Pollution Monitoring System using Wireless Sensor Network \cite{15} }
In this paper they simulated the three air pollutants gases including carbon monoxide, carbon dioxide sulphur dioxide in air. They also apply the approach in various applications like leaking cooking gas in our homes, to alert the workers in oil gas industry to detect the leakage etc. In this they used a sensing unit, a processing unit in the microcontroller, a radio component. The node is designed by integrating the sensor associate circuitry, Atmega 328p low power microcontroller and Xbee communication module. The operating system that runs in the Xbee, coordinates the substances measurement process the acquisition of the change in gas percentages in air and coordinates with the Xbee module for data transmission to the zigbee router. The pollution detector consists array of sensors. 
 Dependence Power consumption of sensor nodes need to be minimized. The selection of sensor and material used in construction of the sensor should select such that there should be minimum changes in the accuracy of the system. In May 2012 V. Ramya and B. Palaniappan worked on the topic Embedded system for Hazardous Gas detection and Alerting by designing a microcontroller based toxic gas detecting and alerting system which sensed gases like LPG and propane and displayed on LCD. Also an alarm was generated and SMS alert were sent to authorized person through the GSM when the level of gases exceeds certain limit.
The system was designed using PIC 16F877 Microcontroller and sensors MQ-2 and MQ-7 for sensing LPG and Propane respectively and displayed on the monitor. When the level of LPG and Propane exceeds a critical level (LPG 1000 ppm and Propane 10000ppm), then an alarm is generated and SMS is sent to the authorized user. But here only two gases are detected (LPG and Propane) but lacks the detection of other harmful pollutant which are present in the environment. Although it is an automated system but it requires to reset after every critical situations.

\subsubsection{Indoor Air Quality in Homes, Offices \& Resturants in Korean Urban Areas—Indoor Outdoor Relationship \cite{16}}
In this paper, Indoor air quality was monitored and measured pollutants were respirable suspended particulate matter (RSP), carbon monoxide (CO), carbon dioxide ($CO_2$), nitrogen dioxide ($NO_2$), and a range of volatile organic compounds (VOCs). In addition, in order to evaluate the effect of smoking on indoor air quality, analyses of parameters associated with environmental tobacco smoke (ETS) were undertaken, which are nicotine, ultraviolet (UVPM), florescence (FPM) and solanesol particulate matter (SolPM). Further both indoor and outdoor air quality were measured and compared and Impact of seasonal differences on both indoor and outdoor air quality were also studied. It was found that indoor pollution in winters was comparatively higher than summer. Indoor Air quality difference due to difference in location were studied and compared. Impact of outdoor pollution on indoor air quality were seen and examined.

\subsubsection{Investigation of Indoor Air Quality at Residential Homes in Hong Kong-case Study \cite{17}}
 Air pollutants measured in this study included carbon dioxide ($CO_2$), respirable suspended particulate matter ($PM_{10}$), formaldehyde (HCHO), volatile organic compounds (VOCs) and air borne bacteria. During the air measurement, indoor temperature, relative humidity and the age of the building were also recorded. A portable Q-Trak monitor (Model 8551, TSI Inc.) was used to monitor the indoor and outdoor $CO_2$ concentrations, temperature and relative humidity. The $CO_2$ analyser equipped with a thermistor and a thin film capacitive sensor is able to detect $CO_2$ based on the mechanism of non-disperse infrared detection. A Dust-Trak air monitor (Model 8520, TSI Inc.) was used to measure $PM_{10}$ concentrations.

\subsubsection{MAQS: A Personalized Mobile Sensing System for Indoor Air Quality Monitoring. \cite{18}}
This paper describes MAQS, a personalized mobile sensing system for IAQ monitoring.
To improve accuracy and energy efficiency, MAQS incorporates three novel techniques:
An accurate temporal n-gram augmented Bayesian room localization method that requires few Wi-Fi fingerprints.
An air exchange rate based IAQ sensing method, which measures general IAQ using only $CO_2$ sensors. 
A zone-based proximity detection method for collaborative sensing, which saves energy and enables data sharing among users.

\subsubsection{Detecting Indoor Air Pollutants and taking safety measures \cite{19}}
A sensor based e-nose is developed to sniff the pollutants presents in indoor environment by [1] to monitor indoor air quality (IAQ) and maintain good IAQ by controlling HVAC system of the room. Also the IAQ monitoring along with the relationship between health and IAQ is discussed but this lacks a proper architecture. The effect of outdoor air and indoor human activity on mass concentration of PM 10, PM 2.5, PM 1 is discussed. It is shown that concentration of carbon dioxide and $PM_{10}$ in domestic kitchen is greater than in living rooms showing that cooking is a major source of pollution. The influence of outdoor air quality on the indoor air quality is discussed. They also tried to provide quantitative information on the levels of potentially important pollutants in three typical environments (homes, offices, and restaurants), to compare indoor and ambient pollution as part of the task of source appointment, and investigate the extent to which certain indoor pollution sources influence the quality of indoor air in urban areas. The above study has confirmed the importance of ambient air quality in determining the quality of air indoors. Lower detection limits and precisions for the methods used in this study are the limitations of this work.

\subsection{Outdoor Air Quality Monitoring}
As there is increase in urbanisation leading to increased volume of traffic, market places, industries etc. the outdoor air quality is deteriorating day by day. Therefore there is a need of continuous air quality monitoring and using that information in the betterment of general public.

\subsubsection{Air Sensing and Alert Generation \cite{20}}
They simulated the three air pollutants gases including carbon monoxide, carbon dioxide and sulphur dioxide in air. They also apply the approach in various applications like leaking cooking gases in homes, to alert the workers in oil gas industry to detect the leakage etc. In this work they used a sensing unit, a processing unit in the microcontroller, a radio component. The node is designed by integrating the sensor associate circuitry, Atmega 328p low power microcontroller and Xbee, coordinates the substances measurement process the acquisition of the change in gas percentages in air and coordinates with the Xbee module for data transmission to the zigbee router. The pollution detector consist array of sensors. But Dependence Power consumption of sensor nodes need to be minimized and the selection of sensor and material used in construction of the sensor should be selected wisely. Similarly in a microcontroller based toxic gas detecting and alerting system was designing sensed gases like LPG and propane and displayed on LCD. Also an alarm was generated and SMS alert were sent to the authorized person through the GSM when the level of gases exceed certain limit. The system was designed using PIC 16F877 Microcontroller and sensors MQ-2 and MQ-7 for sensing LPG and Propane respectively and displayed on the monitor. When the level of the LPG and Propane exceeds a critical level (LPG greater than 1000 ppm and Propane greater than 10000 ppm), then an alarm is generated and SMS is sent to the authorized user. But here only two gases are detected and lacks the detection of other harmful pollutant which are present in the environment. Although it is an automated system but it requires to reset after every critical situation.

\subsubsection{Environment Monitoring and Air Quality Prediction \cite{21}}
We can use the Air Quality information along with some other useful information to predict the air quality depending upon different criteria. Different methods and modelling techniques can be used to predict air quality of any arbitrary location or time. Following are some literature survey using different useful technologies:-
Using a distance decay regression selection strategy
They reported the first attempt to model NO, $NO_2$ and $NO_x$ concentration in Los Angeles using a land use regression (LUR) approach. The LUR was developed as part of a study to examine the impacts of outdoor air pollution on respiratory health in children. The LUR method seeks to predict pollution concentrations at a given site based on surrounding land use, road network, traffic, physical environment and population characteristics using a series of buffers. In this work, NO, $NO_2$ and $NO_x$ concentrations for the LA metropolitan area were modelled using the ADDRESS modelling strategy. The final three prediction models explained 8186 Models. The model provides a relatively easy and feasible way to improve exposure analysis. The influence of slope gradients decreases suggesting that steeper gradient. It concludes that truck routes exerted higher $NO_x$ emissions and had a positive influence on concentration.
Using Machine Learning (semi-supervised learning) in Modelling and Predicting Air Quality
Machine learning and Artificial intelligence is widely used for prediction and classification purpose. Since Air quality depends upon many parameters in a non-linear way hence prediction work in regard to environmental monitoring needs to deal with large volume of data. Therefore, Neural Networks are best suited for this purpose. They use linear chain conditional random field (CRF) and Artificial neural network to infer the real-time and fine-grained air quality information throughout the city, based on the (historical and real-time) air quality data reported by existing monitor stations and a variety of data sources observed in the city, such as meteorology, traffic flow, human mobility, structure of road networks, and point of interests (POIs). It proposes a semi-supervised learning approach based on a co-training framework that consist of two separated classifiers. One is a spatial classifier based on Artificial Neural Network (ANN), which takes spatially-related features as input to model the spatial correlation between air qualities and different locations. The other is a temporal classifier based on a linear-chain conditional random field (CRF), involving temporally-related features to model the temporal dependency of air quality in a location.
The result show the advantage of this method over four categories of baselines, including linear/Gaussian interpolations, classical dispersion models, well-known classification models like decision tree and CRF, and ANN. The city was divided into disjointed grids assuming that air quality in a grid is uniform. Each has a geospatial coordinate and a set of AQI labels to be inferred or already associated if having an air quality monitor station located.

\subsubsection{Inferring Air Quality and location by using semi-supervised inference model Based on Urban Big Data Technology \cite{22}}
It is very complicated question to find out the most optimum location for placing the sensor to cover a large area and predict the air quality of whole area accurately. They tried two answer two questions in their work. First, to infer real-time air quality of any arbitrary location given environmental data and historical air quality data from very sparse monitoring locations. Second, if one needs to establish few new monitoring stations to improve the inference quality, and to determine the best locations for such purpose.
\\
\\
Here they designed a semi-supervised inference model utilizing existing monitoring data together with heterogeneous city dynamics, including meteorology, human mobility, structure of road networks and point of interests (POIs). It also proposes an entropy minimization model to suggest the best locations to establish new monitoring stations. Evaluation of the proposed approach using Beijing air quality data was done. They divided geo-spatial area into disjoint grids, which becomes the basic unit in inference. The AQI values of most grids were completely unknown while the historical AQI values of a small amount of grids can be obtained through existing monitoring stations.

\subsubsection{Mobile Environment Monitoring \cite{23}}
Small Environment Monitoring boxes dynamically moving around the city or given area can be much more efficient and feasible way to cover a large area for air quality monitoring. It can be much more effective method to gather data of more locations which in turn make the prediction process more realistic and accurate. Thus, designing an online GPRS Sensors Array for air pollution monitoring system can be done for this purpose. The system integrates a single chip micro controller, several air pollution sensors, a GPRS modem and a GPS module. The unit can be placed on the top of any moving device such as public transport vehicle. While the vehicle is on the move, the micro controller generates a frame consisting of the acquired air pollutant level from the sensors array and the physical location that is reported from the attached GPS module. The pollutants frame is then uploaded to the General Packet Radio Service Modem (GPRS-modem) and transmitted to the pollution-server for storing the pollutants level of further usage by interested clients such as production agencies, vehicles regeneration authorities, tourist and insurance companies. The pollution –server is interfaced to Google maps to display real-time pollutants levels and their locations. 
\\
\\
The system software architecture is divided into two layers structure i.e; physical layer and application layer. Physical layer is responsible for acquiring the real time data from the sensor-array and physical location, time and date of the sampled pollutants from the GPS module and is implemented using ANSI C language which is compiled to native microcontroller code. The application layer consist of three primary module: Socket-Server, Air Pollution Index and Google-Mapper. Socket-Server collects and stores data from all the mobile-DAQs. Air Pollution Index calculates pollution categories based on local pollution policies and regulations. But the limitation is that the data collected is limited to the vicinity of six monitoring stations. Also this system monitors only three pollutants that is CO, $NO_2$ and $SO_2$.

\subsubsection{Comparison of different approaches of Air Quality Prediction \cite{24}}
A comprehensive comparison between different prediction approaches gives us the idea to select best approach to proceed in a systematic way. In prediction of pollutants $PM_{10}$ and Ozone has been taken into consideration. Here feed forward neural networks (FFNNs), recognized as state-of-the-art approach for statistical prediction of air quality, and are compared with two alternative approaches derived from machine learning: pruned neural networks (PNNs) and lazy learning (LL). All the three approaches are tested in the prediction of ozone and $PM_{10}$ and the predictors are trained. The prediction, issued at 9 a.m. for the current day, show a satisfactory reliability. LL provides the best performances on indicators related to average goodness of the prediction, while PNNs are superior to the other approaches in detecting of the exceedances of alarm and attention thresholds. The better outcome of all the approaches on $PM_{10}$ with respect to ozone can be due to daily average prediction target, which generates a smoother time series than the maximum 8-h moving average adopted for ozone.

\subsubsection{A Mobile GPRS-Sensors Array for Air Pollution Monitoring \cite{25}}
Designing an online GPRS-Sensors Array for air pollution monitoring system can be done. The system integrates a single chip micro controller, several air pollution sensors, a GPRS modem and a GPS module. The unit can be placed on the top of any moving device such as a public transportation vehicle. While the vehicle is on the move, the micro controller generates a frame consisting of the acquired air pollutant level from the sensors array and the physical location that is reported from the attached GPS module. The pollutants frame is then uploaded to the General Packet Radio Service Modem (GPRS-Modem) and transmitted to the Pollution-Server via the public mobile network. A database server is attached to the Pollution-Server for storing the pollutants level for further usage by interested clients such as environment production agencies, vehicles regeneration authorities, tourist and insurance companies. The pollution-Server is interfaced to Google maps to display real-time pollutants levels and their locations. The system software architecture is divided into two layers structure i.e; physical layer and application layer. Physical layer is responsible for acquiring the real- time data from the sensors-array and the physical location, time and date of the sampled pollutants from the GPS module and is implemented using ANSI C language which is compiled to native microcontroller code. The application layer consists of three primary modules: Socket-Server, Air-Pollution-Index, and Google-Mapper. Socket-Server collects and stores pollutant data from all the Mobile-DAQs. Air Pollution-Index calculates pollution categories based on local pollution policies and regulations. Finally, Google-Mapper, makes this pollution information available over the Internet. But the limitations are that the data collected is limited to the vicinity of the six monitoring stations. Also this system monitors only three pollutants that is CO, $NO_2$ and $SO_2$.

\subsubsection{Inferring Air Quality Based on Urban Big Data \cite{26}} 
They tried to answer two questions in their work. First, to infer real-time air quality of any arbitrary location given environmental data and historical air quality data from very sparse monitoring locations. Second, if one needs to establish few new monitoring stations to improve the inference quality, how to determine the best locations for such purpose? Here they designed a semi-supervised inference model utilizing existing monitoring data together with heterogeneous city dynamics, including meteorology, human mobility, 5 structure of road networks, and point of interests (POIs). It also proposes an entropy minimization model to suggest the best locations to establish new monitoring stations. Evaluation of the proposed approach using Beijing air quality data was done. They divide geo-spatial area into disjointed grids, which be-comes the basic unit or instance in the inference. Each grid, denoted by r, is a 1km*1km sub-area, with its own geographical coordination. Each grid is associated with an AQI value, of which some need to be inferred. The AQI values of most grids were completely unknown while the historical AQI values of a small amount of grids can be obtained through existing monitoring stations. The meteorology, road network, and POI information of each grid are assumed to be available. The goal was to infer the AQI distribution of any unobserved location v at any given time stamp t(i). The proposed algorithm consists of four stages. In the first stage construction of a spatial-temporal graph, the AQI Affinity Graph (AG) was done to model the spatial-temporal correlation between grids. In the second stage they try to learn the weights of the edges, assuming they represent the correlations between nodes based on their features. The third stage emphasizes on inferring the AQI values for locations. In this stage model presumes those grids whose features were close to each other tend to share similar AQI values. In the final stage the feature weights are adjusted to minimize the uncertainty of the model on inferring the unobserved locations. Finally to recommend the most proper locations in which building new air-quality monitoring stations can lead to the largest accuracy improvement on air quality inference, they proposed entropy-minimization greedy technique which tries to identify a set of nodes that are uncorrelated with the more confident (i.e. low entropy) ones most of the time as the recommended locations for deployment. It is much more effective than myopically minimize entropy or other heuristics. Efficiency of this model could be increased through parallelization.
