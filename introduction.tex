\chapter{EVOLUTIONARY ALGORITHMS}

\section{Introduction to EA}In artificial intelligence, an evolutionary algorithm (EA) is a subset of evolutionary computation, a generic population-based optimization algorithm. An EA uses mechanisms inspired by biological evolution, such as reproduction, mutation, recombination, and selection. Candidate solutions to the optimization problem play the role of individuals in a population, and the fitness function determines the quality of the solutions (see also loss function). Evolution of the population then takes place after the repeated application of the above operators.In details, the underlying idea behind the algorithm is:\\
Given a population of individuals the environmental pressure causes natural selection (survival of the fittest) and this causes a rise in the fitness of the population.Given a quality function to be maximised we can randomly create a set of candidate solutions, i.e., elements of the function’s domain, and apply the quality function as an abstract fitness measure- the higher the better. Based on this fitness, some of the better candidates are chosen to seed the next generation by applying recombination and/or mutation to them.\\
Recombination is an operator applied to two or more selected candidates (the so-called parents) and results one or more new candidates (the children). Mutation is applied to one candidate and results in one new candidate.Executing recombination and mutation leads to a set of new candidates date (the 	offspring) that compete-based on their fitness – with the old ones for a place in the 	next generation. This process can be iterated until a candidate with sufficient quality (a solution) is found or a previously set computational limit is reached.\\
\section{General algorithm of EA.}BEGIN\\
        INITIALISE population with random candidate solutions;\\
		EVALUATE each candidate’s fitness;\\
		REPEAT UNTIL (TERMINATION CONDITION is satisfied) DO\\\
           1.	SELECT parents;\\
           2.	RECOMBINE pairs of parents;\\
           3.	MUTATE the resulting offspring;\\
           4.	EVALUATE new candidates;\\
           5.	SELECT individuals for the next generation;\\
        END\\\\
\includegraphics[width=1\textwidth]{./ge}\\[1cm]
\section{Some evolutionary algorithms.}
\subsection{Ant colony optimization.}Ant colony optimization algorithm(ACO) is a probabilistic technique for solving computational problems which can be reduced to finding good paths through graphs.This algorithm is a member of the ant colony algorithms family, in swarm intelligence methods, and it constitutes some optimizations. Initially proposed by Marco Dorigo in 1992 in his PhD thesis, the first algorithm was aiming to search for an optimal path in a graph, based on the behavior of ants seeking a path between their colony and a source of food. The original idea has since diversified to solve a wider class of numerical problems, and as a result, several problems have emerged, drawing on various aspects of the behavior of ants.\\
\includegraphics[width=0.5\textwidth]{./ant}\\[1cm]
\subsection{Genetic algorithm.}Genetic algorithm (GA) is a search heuristic that mimics the process of natural selection. Genetic algorithms belong to the larger class of evolutionary algorithms (EA), which generate solutions to optimization problems using techniques inspired by natural evolution, such as inheritance, mutation, selection, and crossover.Next step is to generate a second generation population of solutions from those selected through a combination of genetic operators: crossover (also called recombination), and mutation.\\
For each new solution to be produced, a pair of "parent" solutions is selected for 	breeding from the pool selected previously.By producing a "child" solution using 	the above methods of crossover and mutation, a new solution is created which typically shares many of the characteristics of its "parents". New parents are selected for each new child, and the process continues until a new population of solutions of appropriate size is generated. Although reproduction methods that are based on the use of two parents are more "biology inspired",quality chromosomes.\\
These processes ultimately result in the next generation population of 	chromosomes that is different from the initial generation.Generally the average fitness will have increased by this procedure for the population, since only the best organisms from the first generation are selected for breeding, along with a small proportion of less fit solutions. These less fit solutions ensure genetic diversity within the genetic pool of the parents and therefore ensure the genetic diversity of the subsequent generation of children.\\
\includegraphics[width=0.5\textwidth]{./ee}\\[1cm]
\subsection{Particle swarm optimization.}Particle swarm optimization is a heuristic global optimization method and also an optimization algorithm, which is based on swarm intelligence. It comes from the research on the bird and fish flock movement behaviour.These particles are moved around in the search-space according to a few simple formulae. The movements of the particles are guided by their own best known position in the search-space as well as the entire swarm's best known position. When improved positions are being discovered these will then come to guide the movements of the swarm. The process is repeated and by doing so it is hoped, but not guaranteed, that a satisfactory solution will eventually be discovered.\\\\
\includegraphics[width=0.5\textwidth]{./pso}\\[1cm]

