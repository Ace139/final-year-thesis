\chapter{BACKGROUND STUDY}
\section{What is an algorithm?}An algorithm is a process or set of rules to be followed in calculations or other problem-solving operations, especially by a computer.Broadly,it is a step-by-step procedure for solving a problem or accomplishing some end especially by a computer.\\
\section{Classification of algorithms.}Based on various criteria, algorithms are divided into:-\\
ON THE BASIS OF SEQUENCE OF THE INTERNAL EXECUTIONS\\
\subsection{Deterministic algorithms.}Deterministic algorithms can be defined in terms of a state machine: a state	describes what a machine is doing at a particular instant in time.If the machine is deterministic, this means that from this point onwards, its current state determines what its next state will be its course through the set of states is predetermined.This type of algorithms are such which, given a particular input, will always 	produce the same output, with the underlying machine always passing through the same sequence of states.In simple words, deterministic algorithms produce on a given input the same results following the same computation steps.\\
Although real programs are rarely purely deterministic, it is easier for humans as 	well as other programs to reason about programs that are.To solve real time problems if we use deterministic algorithm its time complexity increases exponentially when dimension increases.Example - if you are sorting elements that are strictly ordered(no equal elements) the output is well defined and so the algorithm is deterministic.\\
\subsection{Non-Deterministic algorithms.}A randomized or non-deterministic algorithm is an algorithm that, even for the same input, can exhibit different behaviors on different runs. It is different from its more familiar deterministic counterpart in its ability to arrive at outcomes using various routes. In other words, a non-deterministic algorithm represents a single path stemming into many paths, some of which may arrive at the same output and some of which may arrive at unique outputs.
Example – Evolutionary algorithms.\\
In computational complexity theory, non-deterministic algorithms are ones that, at every possible step, can allow for multiple continuations (imagine a man walking down a path in a forest and, every time he steps further, he must pick which fork in the road he wishes to take). These algorithms do not arrive at a solution for every possible computational path; however, they are guaranteed to arrive at a correct solution for some path (i.e., the man walking through the forest may only find his cabin if he picks some combination of "correct" paths).\\
TYPES OF RANDOMIZED ALGORITHMS:-\\
1. \textbf{Las Vegas type algorithms}\\
A Las Vegas algorithm will always produce the same result on a given input.Randomization will only affect the order of the internal executions.Example- Quick sort\\
2.\textbf{ Monte Carlo type algorithms}\\
In the case of Monte Carlo algorithms, the result may might change, even be wrong.However, a Monte Carlo algorithm will produce the correct result with a certain probability. Example- PSO\\
\includegraphics[width=0.8\textwidth]{./pic}\\[1cm]
ON THE BASIS OF availability OF INPUT:-\\
\subsection{Online algorithms.}Online algorithms are algorithms that do not know their input at the beginning; it is 	given to them online. Such algorithms are DYNAMIC in nature.Example - ski problem. A skier must decide every day she goes skiing, whether to rent or to buy skis, unless or until she decides to buy them. The skier does not know how many days she can ski, because the weather is unpredictable.\\
\subsection{Offline algorithms.}Offline algorithms are algorithms know their input at the beginning, i.e. input is predetermined.Such algorithms are STATIC in nature.\\
\section{Types of problems.}The total defined problem sphere has been divided into four classes - \textbf{P, NP, NP-complete and NP-Hard problems}.\\
\subsection{ P and NP class.}\textbf{P} is set of problems that can be solved by a deterministic algorithm in Polynomial time.\\
\textbf{NP} is set of decision problems that can be solved by a non-deterministic algorithm in Polynomial time.P is subset of NP (any problem that can be solved by deterministic machine in polynomial time can also be solved by non-deterministic machine in polynomial time).\\
\subsection{ NP-complete.}\textbf{NP-complete} problems are the hardest problems in NP set. A decision problem is NP-complete if:\\
1) L is in NP (Any given solution for NP-complete problems can be verified quickly, but there is no efficient known solution).\\
2) Every problem in NP is reducible to L in polynomial time.\\
\subsection{NP-hard.}\textbf{NP-Hard }- A problem is NP-Hard if it follows property 2 mentioned above, doesn’t need to follow property 1.This means a problem K is NP-Hard if every problem in NP is reducible to K in polynomial time but the problem may or may not be in NP. Therefore, NP-Complete set is also a subset of NP-Hard set.\\
\section{Why non-deterministic algorithm?}In algorithm design, non-deterministic algorithms are often used when the problem solved by the algorithm inherently allows multiple outcomes (or when there is a single outcome with multiple paths by which the outcome may be discovered, each equally preferable.
As we know that increasing dimension or input size increases the complexity of problem and using a deterministic algorithm under such situations, increases the time complexity exponentially. So, we use non-deterministic algorithms for such problems. The class of NP-complete and NP-hard problems falls under this category. Such problems can be solved using a deterministic algorithm, but then it would require huge amount of time, may be a day, an year or even more,which is not feasible in real-time.So,we use some non-determinism i.e. randomicity while solving such problems.Hence, to solve problems corresponding to NP-complete and NP-hard class we use non-deterministic algorithms.\\