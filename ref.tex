\cleardoublepage
%\pagebreak
\phantomsection
\addcontentsline{toc}{chapter}{References}
\begin{thebibliography}{99}

\bibitem{citation-1-wikipedia}wikipedia,\ \url{<wikipedia.org/wiki/Pollution>}

\bibitem{citation-2-}Durgapur adda,\ \url{<Durgapuradda.com>}

\bibitem{citation-3-}Tata steel discharge in Kharki river Jamshedpur,\ \url{<Stained steel>}
\bibitem{citation-4-}Durgapur coalmine in chandrapur,\ \url{<wordpress.com>}
\bibitem{citation-5-}Noise within limits,\ \url{google.com}
\bibitem{citation-6-}2 know about,\ \url{google.com}
\bibitem{citation-7-}Mortality linked to air pollution,\ \url{atlas.com}
\bibitem{citation-8-}pollution disasters,\ \url{<www.allianz.com}
\bibitem{citation-9-}Pollution breakpoint,\ \url{<Datadrivenyale.com>}
\bibitem{citation-10-}World Health Organisation,\ \url{who.int}
\bibitem{citation-11-}Greenpeace,\ \url{www.greenpeace.org}
\bibitem{citation-12-}Deadly Air Pollution,\ \url{who.int}
\bibitem{citation-13-}The Economic consequences of Air Pollution,\ \url{<oecd.org>}
\bibitem{citation-14-}Time of India,\ \url{<TimesofIndia.Indiatimes.com>}
\bibitem{citation-15-}Khedo, Kavi K., Rajiv Perseedoss, and Avinash Mungur. \textit{"A wireless sensor network air pollution monitoring system." arXiv preprint arXiv:1005.1737 (2010)}.\ \url{}
%\bibitem{citation-16-}Indoor Air Quality in Homes Offices & Resturants in Korean Urban Areas Indoor Outdoor Relationship\ \url{}
\bibitem{citation-16-}Baek, Sung-Ok, Yoon-Shin Kim, and Roger Perry. "Indoor air quality in homes, offices and restaurants in Korean urban areas—indoor/outdoor relationships.\textit{" Atmospheric Environment 31.4 (1997): 529-544}.
\ \url{}

\bibitem{citation-17-}Lee, Shun Cheng, Wai-Ming Li, and Chio-Hang Ao. "Investigation of indoor air quality at residential homes in Hong Kong—case study.\textit{" Atmospheric Environment 36.2 (2002): 225-237}. 

\bibitem{citation-18-}Jiang, Yifei, et al. "MAQS: a personalized mobile sensing system for indoor air quality monitoring.\textit{" Proceedings of the 13th international conference on Ubiquitous computing ACM, 2011}.\ \url{}
\bibitem{citation-19-}Detecting Indoor Air Pollutants and taking safety measures\ \url{}

\bibitem{citation-20-}Air Sensing and Alert Generation\ \url{}
\bibitem{citation-21-}Environment Monitoring and Air Quality Prediction\ \url{}

\bibitem{citation-22-}Inferring Air Quality and location by using semi-supervised inference model Based on Urban Big Data Technology\ \url{}
\bibitem{citation-23-}Mobile Environment Monitoring\ \url{}

\bibitem{citation-24-}Corani, Giorgio. "Air quality prediction in Milan: feed-forward neural networks, pruned neural networks and lazy learning.\textit{" Ecological Modelling 185.2 (2005): 513-529.}\ \url{}
\bibitem{citation-25-}Al-Ali, A. R., Imran Zualkernan, and Fadi Aloul. "A mobile GPRS-sensors array for air pollution monitoring.\textit{" IEEE Sensors Journal 10.10 (2010): 1666-1671.}\ \url{}
\bibitem{citation-26-}Zheng, Yu, Furui Liu, and Hsun-Ping Hsieh. "U-air: When urban air quality inference meets big data.\textit{" Proceedings of the 19th ACM SIGKDD international conference on Knowledge discovery and data mining. ACM, 2013}.
\bibitem{citation-27-}Air Quality Index(AQI)\ \url{}
\bibitem{citation-28-}Oil and Gas spill\ \url{ http://response.restoration.noaa.gov}
 \bibitem{citation-29-}History Of Pollution\ \url{ https://www.pollutionpollution.com}
 \bibitem{citation-30-}\ \url{ https://www.epa.gov}
 \bibitem{citation-31-}Bibek Poddar, Soumyo Dey et al.On Detecting Acceptable Air Contamination in Classrooms using Low Cost Sensors,\textit{ WACI Workshop, COMSNETS 2017}\\
 

\end{thebibliography}
